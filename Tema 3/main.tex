\documentclass{article}
\usepackage{graphicx} % Required for inserting images
\usepackage{indentfirst}

% Set page size and margins
% Replace `letterpaper' with `a4paper' for UK/EU standard size
\usepackage[letterpaper,top=2cm,bottom=2cm,left=3cm,right=3cm,marginparwidth=1.75cm]{geometry}

\title{Tema 2 - Algoritmica Grafurilor}

\author{Mihail Popovici, Luca Petrovici, Alexandru-Constantin Iov, George-Razvan Rusu}
\begin{document}
\maketitle

\section*{\fontsize{20}{50}\selectfont Problema 1 : Popovici Mihail}

{\fontsize{14}{16}\selectfont 
$\mathbf{Implicatia}$ $\mathbf{directa: (=>)}$ 
\par Dacă $G$ are cuplaj perfect, atunci $\alpha(G) = |G|/2$
\\
\par Un cuplaj perfect în $G$ este un cuplaj $M$ cu proprietatea $|M| = |G|/2$ (fiecare nod din $G$ este incident cu exact o muchie din cuplaj). Astfel, dacă $G$ are cuplaj perfect, fiecare nod al său este inclus în acest cuplaj. 
\par Pe de altă parte, într-un graf bipartit, numărul total de noduri $|G|$ este suma nodurilor din cele 2 părți ale partiției ($|G| = |S| + |T|$, unde $S$ și $T$ sunt mulțimile disjuncte ale nodurilor).
\par Având în vedere că $G$ are cuplaj perfect, fiecare muchie a acestui cuplaj formează o pereche, unul din nodurile perechii fiind din $S$, iar celălalt din $T$, deci numărul total de muchii în cuplaj este $|G|/2$.

\bigskip
$\mathbf{Implicatia}$ $\mathbf{inversa: (<=)}$ 
\par Dacă $\alpha(G) = |G|/2$, atunci $G$ are un cuplaj perfect.
\\

\par Pornind de la ipoteză ($\alpha(G) = |G|/2$), înseamnă că avem un cuplaj maxim $M$ cu $|G|/2$ muchii, iar fiecare muchie din acest cuplaj acoperă noduri distincte. Deoarece $\alpha(G) = |G|/2$, acest cuplaj acoperă toate cele $|G|$ noduri $=>$ $M$ este cuplaj perfect, toate nodurile fiind acoperite de exact o muchie. 

}
\end{document}