\documentclass{article}
\usepackage{graphicx} % Required for inserting images
\usepackage{indentfirst}
\usepackage{amssymb}
\usepackage{amsmath}

% Set page size and margins
% Replace `letterpaper' with `a4paper' for UK/EU standard size
\usepackage[letterpaper,top=2cm,bottom=2cm,left=3cm,right=3cm,marginparwidth=1.75cm]{geometry}

\title{Tema 3 - Algoritmica Grafurilor}

\author{Mihail Popovici, Luca Petrovici, Alexandru-Constantin Iov, George-Razvan Rusu}
\begin{document}
\maketitle

\section*{\fontsize{20}{50}\selectfont Problema 1 : Popovici Mihail}

{\fontsize{14}{16}\selectfont 
$\mathbf{Implicatia}$ $\mathbf{directa: (=>)}$ 
\par Dacă $G$ are cuplaj perfect, atunci $\alpha(G) = |G|/2$
\\
\par Un cuplaj perfect în $G$ este un cuplaj $M$ cu proprietatea $|M| = |G|/2$ (fiecare nod din $G$ este incident cu exact o muchie din cuplaj). Astfel, dacă $G$ are cuplaj perfect, fiecare nod al său este inclus în acest cuplaj. 
\par Pe de altă parte, într-un graf bipartit, numărul total de noduri $|G|$ este suma nodurilor din cele 2 părți ale partiției ($|G| = |S| + |T|$, unde $S$ și $T$ sunt mulțimile disjuncte ale nodurilor).
\par Având în vedere că $G$ are cuplaj perfect, fiecare muchie a acestui cuplaj formează o pereche, unul din nodurile perechii fiind din $S$, iar celălalt din $T$, deci numărul total de muchii în cuplaj este $|G|/2$.

\bigskip
$\mathbf{Implicatia}$ $\mathbf{inversa: (<=)}$ 
\par Dacă $\alpha(G) = |G|/2$, atunci $G$ are un cuplaj perfect.
\\

\par Pornind de la ipoteză ($\alpha(G) = |G|/2$), înseamnă că avem un cuplaj maxim $M$ cu $|G|/2$ muchii, iar fiecare muchie din acest cuplaj acoperă noduri distincte. Deoarece $\alpha(G) = |G|/2$, acest cuplaj acoperă toate cele $|G|$ noduri $=>$ $M$ este cuplaj perfect, toate nodurile fiind acoperite de exact o muchie. 

}

\section*{Exercițiul 2 : Rusu Răzvan}

{\fontsize{14}{16}\selectfont 

\subsection*{Subpunctul a)}

Cunoaștem, din enunțul problemei, că \( c(p_i p_j) = +\infty \), oricare ar fi \( p_i, p_j \in F \). Este, de asemenea, menționat că tăietura \((S, T)\) are o capacitate finită, \( c(S, T) < +\infty \). 

Asta înseamnă că, dacă un arc \( p_i p_j \in F \) ar traversa tăietura \((S, T)\), capacitatea tăieturii ar deveni infinită. Totuși, asta ar contrazice premisa că \( c(S, T) < +\infty \). Prin urmare, niciun astfel de arc nu poate traversa tăietura, ceea ce implică faptul că, dacă \( p_i \in S \), atunci și \( p_j \) trebuie să aparțină lui \( S \).

\subsection*{Subpunctul b)}

\textbf{Proiectele din \( P^+ \):}  
Pentru fiecare \( p_i \in P^+ \), există un arc \( sp_i \) cu capacitate \( c(sp_i) = \sigma_i \). Dacă \( p_i \in S \), atunci profitul său contribuie la suma profitului total. Dacă \( p_i \notin S \), atunci acest arc \( sp_i \) contribuie la tăietura \( c(S, T) \).

\textbf{Proiectele din \( P^- \):}  
Pentru fiecare \( p_i \in P^- \), există un arc \( p_i t \) cu capacitate \( c(p_i t) = -\sigma_i \). Dacă \( p_i \in T \), atunci profitul negativ al proiectului contribuie la tăietura \( c(S, T) \).

\textbf{Profitul total al proiectelor din \( S \setminus \{s\} \):}  
Este suma profitului proiectelor din \( P^+ \) care se află în \( S \), minus pierderile asociate proiectelor din \( P^- \) care se află în \( S \). Acest profit total este:
\[
\text{Profit}(S \setminus \{s\}) = \sum_{p_i \in P^+} \sigma_i - c(S, T).
\]

\subsection*{Subpunctul c)}

Problema poate fi rezolvată printr-un algoritm de flux maxim aplicat pe rețeaua \( R \).

\textbf{Pașii algoritmului:}
\begin{enumerate}
    \item \textbf{Construcția grafului \( G \):}
    \begin{itemize}
        \item Adăugăm un nod sursă \( s \) și un nod destinație \( t \).
        \item Pentru fiecare proiect \( p_i \in P^+ \), adăugăm un arc \( sp_i \) cu capacitate \( \sigma_i \).
        \item Pentru fiecare proiect \( p_i \in P^- \), adăugăm un arc \( p_i t \) cu capacitate \( -\sigma_i \).
        \item Pentru fiecare dependență \( p_i p_j \in F \), adăugăm un arc \( p_i p_j \) cu capacitate \( +\infty \).
    \end{itemize}
    
    \item Folosim algoritmul Edmonds-Karp pentru a calcula fluxul maxim din \( s \) în \( t \).
    
    \item Identificăm tăietura minimă \((S, T)\) asociată fluxului maxim. Nodurile din \( S \setminus \{s\} \) reprezintă mulțimea \( Q \).
    
    \item Returnăm \( Q \), care conține toate proiectele dorite.
\end{enumerate}

\textbf{Complexitatea algoritmului:}
\begin{itemize}
    \item Construirea grafului: \( O(|P| + |F|) \),
    \item Calcularea fluxului maxim (Edmonds-Karp): \( O(|V| \cdot |E|^2) \), unde \( V = |P| + 2 \) și \( E = |F| + |P| \).
\end{itemize}

}

\section*{\fontsize{20}{50}\selectfont Problema 3 : Alexandru-Constantin Iov}

{\fontsize{14}{16}\selectfont 
Fie $G$ un graf cu $\chi(G) = p$.

\bigskip
a) Fie $G$ un graf cu $\chi(G) = p$. Dacă $G$ nu este deja $p$-minimal-cromatic, atunci, conform definiției, $\exists$ un nod $v$ astfel încât $\chi(G-v) = p$. Eliminăm, pe rând, toate aceste noduri până când ajungem la $H$, subgraf al lui $G$, pentru care nu $\exists$ $v$ astfel încât $\chi(H-v) = p \Rightarrow H$ este $p$-minimal-cromatic. În cazul în care $G$ este deja $p$-minimal-cromatic, demonstrația este evidentă.

\bigskip
b) Presupunem prin reducere la absurd că $\exists$ un nod $v \in G$ cu $d(v) < p-1$ și $G$ este $p$-minimal-cromatic. Dacă $d(v) = p-2$, acesta "forțează" $p-2$ noduri să aibă o colorare diferită de el. Dacă eliminăm nodul $v$ din $G$, atunci, conform definiției unui graf $p$-minimal-cromatic, $\chi(G-v) = p-1$. \par Vecinii fostului nod $v$ pot folosi cel mult $p-2$ culori (sunt $p-2$ vecini și avem la dispoziție $p-1$ culori, deci destule). Atunci când adăugăm înapoi nodul $v$ la graf, însă, acesta va putea fi colorat cu cea de-a $(p-1)$-a culoare. 
\par De aici, va rezulta că $\chi(G) = p-1$. Observăm că am ajuns la o contradicție, deoarece $\chi(G)$ ar trebui să fie egal cu $p \Rightarrow G$ nu este $p$-minimal-cromatic. Prin contrazicerea presupunerii, dacă $\exists$ un nod $v \in G$ cu $d(v) < p-1$, $G$ nu este $p$-minimal-cromatic $\Rightarrow$ Daca G este p-minimal-cromatic, atunci $\delta$ (G) $\ge p - 1$.

\bigskip
c) $\mathbf{Implicatia\ directa\ (=>)}$ Dacă $G$ este $3$-minimal-cromatic atunci $G$ este un circuit impar indus.
\par
Mai întâi, pentru că $\chi(G) = 3$, $G$ trebuie să conțină cel puțin un circuit impar, întrucât acesta este unitatea de bază necesară pentru a $3$-colora un graf. Dacă nu ar conține unul, atunci graful ar fi bipartit, și ar putea fi colorat cu $2$ culori, deci $\chi(G) = 2$.
\par
Pentru a fi minimal, $\chi(G-v) = 2$, adică bipartit, $\forall v \in V(G)$. Din acest lucru, rezultă faptul că $G$ conține un singur circuit impar. Mai mult, acest circuit trebuie, la rândul său, să fie indus. Dacă circuitul nu ar fi indus, atunci ar exista riscul ca graful $G-v$ să conțină la rândul lui un alt circuit impar, deci ar fi tot $3$-colorabil, nu $2$-colorabil, lucru ce contrazice definiția unui graf $p$-minimal-colorabil. 
\par
În concluzie, dacă $G$ este $3$-minimal-cromatic, atunci $G$ este un circuit impar indus.

$\mathbf{Implicatia\ indirecta\ (<=)}$ Daca G este un circuit impar indus, atunci este 3-minimal-cromatic.

Evident, dacă $G$ este un circuit impar indus, acesta va fi colorat cu 3 culori. De exemplu, pentru circuitul $1 - 2 - 3 - 1$, culorile vor fi alb - negru - gri - alb, adică 3, din cauza faptului că ultimul nod este adiacent cu primul. Astfel, este indeplinită prima condiție pentru un graf 3-minimal-cromatic. Dacă scoatem orice nod din circuitul indus, atunci vom obține un lanț de lungime pară, care poate fi colorat doar cu 2 culori, îndeplinind și cea de-a doua condiție pentru un graf 3-minimal-colorabil.

În concluzie, dacă $G$ este un circuit impar indus, atunci este 3-minimal-cromatic.

}

\section*{\fontsize{20}{50}\selectfont Problema 4 : Petrovici Luca}

{\fontsize{14}{16}\selectfont

Pentru a arăta că \textbf{EDGE\_BIP} este NP-completă:

\begin{enumerate}
    \item Arătăm că EDGE\_BIP este în clasa NP, adică un candidat pentru a fi soluție a problemei poate fi verificat în timp polinomial. Acest lucru se poate realiza astfel:
    \begin{enumerate}
        \item adunăm ponderile muchiilor eliminate și comparăm cu $a$ (timp liniar, $O(n)$)
        \item realizăm o parcurgere BFS sau DFS pentru a 2-colora graful ($O(|V| + |E|)$)
    \end{enumerate}
    În cazul în care acea sumă a ponderilor este mai mică sau egală cu $a$ și graful este 2-colorabil, înseamnă că după eliminarea muchiilor selectate a rămas un graf indus bipartit.
    \item Reducem polinomial de la o altă problema NP-completă: MAXCUT $ \leqslant _p$ EDGE\_BIP
    \par \textbf{MAXCUT} (cunoscută ca fiind NP-completă)
    \par \textbf{Instanță:} $G = (V, E)$ un graf, $c : E \rightarrow \mathbb{R}$ o funcție de pondere și $k \in \mathbb{R}$.
    \par \textbf{Întrebare:} există o tăietură în $G$ de pondere $\geq k$?
    \par Considerăm datele unei instanțe MAXCUT și folosindu-ne de acestea vom construi datele unei instanțe EDGE\_BIP.
    \begin{enumerate}
        \item Graful $G$ rămâne nemodificat
        \item Similar, funcția de pondere rămâne neschimbată
        \item Definim $W = \sum_{e \in E} c(e)$. $a$ va fi $W - k$
    \end{enumerate}
    Vom elimina muchiile care \textbf{nu} aparțin tăieturii, pentru a obține un graf bipartit. Ponderea lor va fi ponderea totală din care se scade ponderea muchiilor din tăietură (care sunt acceptate în bipartiție).
    \par Să presupunem că răspunsul la MAXCUT pentru instanța considerată este da. Astfel, există o tăietură de pondere minim $k$ în $G$. Muchiile care nu aparțin tăieturii vor avea ponderea $W - k$ adică $a$, deci prin eliminarea lor se obține un graf bipartit și se satisfac condițiile pentru EDGE\_BIP. Răspunsul pentru EDGE\_BIP la această instanță este și el da.
    \par Să presupunem că răspunsul la MAXCUT pentru instanța considerată este nu. Astfel, nu există o tăietură de pondere minim $k$ în $G$, adică ar putea exista doar tăieturi de ponderi mai mici decăt $k$ ceea ce implică faptul că muchiile ce nu aparțin tăieturii vor avea pondere mai mare decât $a$ (din nou, din relația $W - k = a$). Răspunsul pentru EDGE\_BIP pentru instanța considerată este de asemenea nu.
\end{enumerate}

Prin urmare, din $1.$ si $2.$ rezultă că EDGE\_BIP este NP-completă.

}

\end{document}