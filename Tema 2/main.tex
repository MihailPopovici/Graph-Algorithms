\documentclass{article}
\usepackage{graphicx} % Required for inserting images
\usepackage{indentfirst}

% Set page size and margins
% Replace `letterpaper' with `a4paper' for UK/EU standard size
\usepackage[letterpaper,top=2cm,bottom=2cm,left=3cm,right=3cm,marginparwidth=1.75cm]{geometry}

\title{Tema 2 - Algoritmica Grafurilor}

\author{Mihail Popovici, Luca Petrovici, Alexandru-Constantin Iov, George-Razvan Rusu}
\begin{document}
\maketitle

\section*{\fontsize{20}{50}\selectfont Problema 2}
\subsection*{\fontsize{16}{30}\selectfont Subpunctul a}
{\fontsize{14}{16}\selectfont 
    Pentru a determina un arbore partial de cost maxim in $G$ vom folosi o variatie a algoritmului lui Kruskal. Astfel, in loc de a sorta muchiile in ordine crescatoare a costurilor, le vom ordona in mod descrescator.
    \\ 
    \par sort $E=\lbrace e_1, e_2, ..., e_m \rbrace$ $//$ $a.i.$  $c(e_1)\geq c(e_2)...\geq c(e_m)$

    \par $T \leftarrow \emptyset; i \leftarrow 1;$
    \par $while$  $(i \leq m)$  $do$
    \par \hspace*{1cm} $if$ $(\langle T \cup \lbrace e_i \rbrace \rangle _G$ nu are circuite $)$  $then$
    \par \hspace*{1.5cm} $ T \leftarrow T \cup \lbrace e_i \rbrace ;$
    \par \hspace*{1cm} $i++;$
    }

\subsection*{\fontsize{16}{30}\selectfont Subpunctul b}
{\fontsize{14}{16}\selectfont 
$\mathbf{Implicatia}$ $\mathbf{directa:}$ 
\\
\par Ipoteza : $x$ si $y$ sunt conectate in $G_\alpha$
\par Concluzie : $x$ si $y$ sunt conectate in $T_\alpha$         
\\
\par Din ipoteza, $x$ si $y$ sunt conectate in $G_\alpha$. Deci exista un drum $P_{xy}$ intre $x$ si $y$ in $G_\alpha$, format din muchii $e_1$, $e_2$, ..., $e_k$, unde $c(e_i) \geq \alpha$ pentru fiecare muchie.
Prin definitie, $T$ este conex si contine un set de muchii care asigura conectivitatea nodurilor din $G$, dar selecteaza numai muchii de cost maxim in sensul construirii arborelui. In procesul de construire al lui $T$, orice muchie care nu este inclusa in $T$ fie:
\par 1) Nu este necesara pentru conectivitate, adica exista alte muchii care asigura legatura intre componentele respective
\par 2) Este inlocuita cu o muchie de cost mai mare sau egal, deoarece $T$ maximizeaza costul muchiilor incluse. 
\par Astfel, pentru orice muchie $e_i$ din drumul $P_{xy}$ din $G_\alpha$, daca $e_i$ nu se afla in $T$, trebuie sa existe in $T$ un drum alternativ care conecteaza aceleasi componente si care utilizeaza doar muchii de cost $\geq \alpha$.
\par Deoarece $P_{xy}$ este format exclusiv din muchii cu cost $\geq \alpha$, drumul dintre $x$ si $y$ in $G_\alpha$ este reprodus in $T_\alpha$, intrucat $T_\alpha$ contine suficiente muchii pentru a conserva conectivitatea nodurilor din $T$. 
\par In concluzie, daca $x$ si $y$ fac parte din aceeasi componenta conexa in $G_\alpha$, atunci apartin aceleiasi componente conexe in $T_\alpha$.    
\\
\par $\mathbf{Implicatia}$ $\mathbf{inversa:}$
\\
\par Ipoteza : $x$ si $y$ sunt conectate in $T_\alpha$
\par Concluzie : $x$ si $y$ sunt conectate in $G_\alpha$
\\
\par $T$ este un arbore partial al lui $G$, deci $T$ conecteaza toate nodurile din $G$ si este un subgraf al lui $G$. Fiindca $T_\alpha$ pastreaza doar muchiile din $T$ cu $c(e)\geq \alpha$, iar $G_\alpha$ contine toate muchiile cu aceasta proprietate, toate muchiile din $T_\alpha$ sunt incluse si in $G_\alpha$. Deoarece $T_\alpha$ este subarbore al lui $T$ si $G_\alpha$ este subgraf al lui $G$, toate conexiunile intre nodurile din $T_\alpha$ sunt valabile si in $G_\alpha$.
\\
}    

\subsection*{\fontsize{16}{30}\selectfont Subpunctul c}
{\fontsize{14}{16}\selectfont Initial vom folosi algoritmul lui Kruskal modificat de la subpunctul a$)$ pentru a construi arborele partial de cost maxim, $T$, din $G$, urmand sa parcurgem drumul de la $s$ la $t$ astfel:
    \\
    \par $vec \leftarrow \emptyset;$ 
    \par $for$  $((u,v)\in T)$ $do$
    \par \hspace*{1cm} //$construim$  $listele$  $de$  $adiacenta$
    \par \hspace*{1cm} $add$  $u$  $to$  $vec[v];$
    \par \hspace*{1cm} $add$  $v$  $to$  $vec[u];$
    \\
    \par //$definim$  $functia$  $DFS$
    \par $method$ $DFS(current\_ node, last\_node, target, path):$
    \par \hspace*{1cm} $if$ $(current\_node == target)$ $then$
    \par \hspace*{1.5cm} $return$  $true;$
    \par \hspace*{1cm}$for$  $(v \in vec[current\_node])$  $do$
    \par \hspace*{1.5cm} $if$ $(v != last\_node)$  $then$
    \par \hspace*{2cm} $path \leftarrow path \cup (current\_node, v)$
    \par \hspace*{2cm} $if$  $(DFS(v, current\_node, target, path) == true)$ $then$
    \par \hspace*{2.5cm} $return$ $true;$ 
    \par \hspace*{2cm} $else$
    \par \hspace*{2.5cm} $path \leftarrow path$ $\setminus$ $(current\_node, v)$
    \par \hspace*{1cm} $return$ $false;$
    \\
    \par $path \leftarrow \emptyset;$
    \par $DFS(s, NONE, t, path);$
    \\
    \par //$parcurgem$ $muchiile$ $din$ $path$ $si$ 
    \par //$o$ $alegem$ $pe$ $cea$ $cu$ $costul$ $cel$ $mai$ $mic$
    \par $min \leftarrow \infty$
    \par $for$  $(e\in path)$ $do$
    \par \hspace*{1cm} $if$ $(c(e)<min)$ $then$
    \par \hspace*{1.5cm} $min \leftarrow c(e)$
    \par //astfel, cantitatea maxima dintr-un produs care poate
    \par //fi transportat de la s la t se gaseste in 
    \par //variabila min 

}

\section*{\fontsize{20}{50}\selectfont Problema 3}
\subsection*{\fontsize{16}{30}\selectfont Subpunctul a}
{\fontsize{14}{16}\selectfont 

\centerline {$ c(T|e) = \sum_{ e'\in E(T)\setminus \lbrace e \rbrace } c(e') = \sum_{ e'\in E(T) } c(e') - c(e) = c(T) - c(e)$ }
}

\subsection*{\fontsize{16}{30}\selectfont Subpunctul b}
{\fontsize{14}{16}\selectfont 
    Fie $T$ un arbore partial de cost minim al lui $G$, avand costul minim $c(T)$. Conform subpuctului $a$, prin contractarea muchiei e, obtinem $c(T|e) = c(T) - c(e)$.
    Sa presupunem prin reducere la absurd ca $T|e$ $\mathbf{N} \mathbf{U}$ este un arbore partial de cost minim al lui $G|e$. Atunci ar exista un alt arbore partial $T'|e$ a lui $G|e$, cu $c(T'|e) < c(T|e)$.
    \par Daca $T'|e$ este un arbore de cost mai mic n $G|e$, atunci putem extinde acest arbore n $G$ adaugand inapoi muchia $e$ pentru a obtine un arbore $T'$ al lui $G$. Costul acestui arbore $T'$ ar fi:
    \par \centerline { $c(T') = c(T'|e) + c(e)$ }
    \par Deoarece $c(T'|e) < c(T|e)$, rezulta ca:
    \par \centerline {$c(T'|e) < c(T|e) + c(e) = c(T) $}
    \par Aceasta contrazice ipoteza ca $T$ este un arbore partial de cost minm al lui $G$. Prin urmare, $T|e$ trebuie sa fie un arbore partial de cost minim al lui $G|e$.
    }

\section*{\fontsize{20}{50}\selectfont Problema 4}
\subsection*{\fontsize{16}{30}\selectfont Subpunctul a}
{\fontsize{14}{16}\selectfont
$\mathbf{Implicatia}$ $\mathbf{directa:}$
\\
\par Ipoteza: $G$ are o d-tăietură nevidă
\par Concluzia: $G$ nu este tare conex
\\
\par Știm din ipoteză ca $G$ are o d-taietură nevidă, deci există o submulțime de arce $F \subseteq E$ cu proprietatea că există o bipartiție $(A, B)$ a lui $V (A, B \neq \emptyset)$ așa încât $F = \lbrace uv, \in E : u \in A, v \in B\rbrace$ și $\lbrace vu, \in E : u \in A, v \in B \rbrace = \emptyset$. Deci, există cel puțin două noduri $x \in A$ și $y \in B$ cu arcul $xy$ și niciun alt arc din $B$ în $A$. Acest lucru permite parcurgerea din $x$ în $y$, dar nu și invers, deci $G$ nu este tare conex.
\\
\par $\mathbf{Implicatia}$ $\mathbf{inversa:}$
\\
\par Ipoteza: $G$ nu este tare conex
\par Concluzia: $G$ are o d-tăietură nevidă
\\
\par Știm din ipoteză ca $G$ nu este tare conex, adică $G$ conține cel puțin două noduri $x$ și $y$ între care există arcul $xy$ și nu există niciun $yx$-drum. Considerăm mulțimile $A = \lbrace u : \exists uy$-drum în $ G\rbrace \cup \lbrace x \rbrace$ și $B = \lbrace v : \exists yv$-drum în $ G\rbrace \cup \lbrace y \rbrace$ (fiindcă $G$ este slab conex, știm că $A \cup B = V$ și $A \cap B = \emptyset$). În acest fel, vor exista doar arce de la $A$ la $B$ (printre care și $xy$) și niciun arc de la $B$ la $A$, adică $G$ conține măcar d-tăietura nevidă descrisă de mulțimile $A$ și $B$.

\subsection*{\fontsize{16}{30}\selectfont Subpunctul b}
Demonstrăm că (i) este echivalent cu (ii)

Demonstrăm că (i) este echivalent cu (iii). 

(i) $\rightarrow$ (iii): Prin definiție, o d-tăietură partiționează digraful în două submulțimi de noduri ($A$ și $B$) cu proprietatea că există $xy$-drumuri, dar niciun $yx$-drum ($x \in A, y \in B$). Am putea spune că d-tăieturile descriu partiționări ale digrafului în care parcurgerea se face într-un singur sens. Din faptul că $J$ este un d-separator rezultă că $J$ conține măcar o muchie din fiecare d-tăietură. Adăugarea arcelor inverse celor din $J$ ar permite parcurgerea în ambele sensuri între mulțimile oricărei partiționări ale digrafului, așadar, digraful ar deveni tare conex.

(iii) $\rightarrow$ (i): Presupunem prin reducere la absurd că prin adăugarea tuturor arcelor inverse celor din $J$ la $G$ se obține un digraf tare conex, iar $J$ nu este un d-separator. Deoarece $J$ nu este un d-separator, înseamnă că $\exists F$ d-taietură a.î. $F \cap J = \emptyset$. Din asta, rezultă că bipartiția $(A, B)$, conectată de arcele din $F$ va conține $xy$-drumuri, dar niciun $yx$-drum ($x \in A, y \in B$), deci, $G$ nu este tare-conex.  Am ajuns la o contradicție, deci dacă prin adăugarea tuturor arcelor inverse celor din $J$ la $G$ se obține un digraf tare conex, atunci $J$ este un d-separator.

\subsection*{\fontsize{16}{30}\selectfont Subpunctul c}
Fie $\mathcal{F}$ o familie de cardinal maxim (pe care îl notăm cu $p$) de d-tăieturi disjuncte două cate două. Prin definiție, un d-separator intersectează toate d-tăieturile digrafului, așadar d-separatorul va avea cardinalul minim egal cu $p$ (în cazul în care intersectează exact o singură muchie din fiecare d-tăietură din $\mathcal{F}$).

\end{document}