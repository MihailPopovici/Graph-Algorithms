\documentclass{article}
\usepackage{graphicx} % Required for inserting images
\usepackage{indentfirst}

% Set page size and margins
% Replace `letterpaper' with `a4paper' for UK/EU standard size
\usepackage[letterpaper,top=2cm,bottom=2cm,left=3cm,right=3cm,marginparwidth=1.75cm]{geometry}

\title{Tema 2 - Algoritmica Grafurilor}

\author{Mihail Popovici, Luca Petrovici, Alexandru-Constantin Iov, George-Razvan Rusu}
\begin{document}
\maketitle

\section*{\fontsize{20}{50}\selectfont Problema 2}
\subsection*{\fontsize{16}{30}\selectfont Subpunctul a}
{\fontsize{14}{16}\selectfont 
    Pentru a determina un arbore partial de cost maxim in $G$ vom folosi o variatie a algoritmului lui Kruskal. Astfel, in loc de a sorta muchiile in ordine crescatoare a costurilor, le vom ordona in mod descrescator.
    \\ 
    \par sort $E=\lbrace e_1, e_2, ..., e_m \rbrace$ $//$ $a.i.$  $c(e_1)\geq c(e_2)...\geq c(e_m)$

    \par $T \leftarrow \emptyset; i \leftarrow 1;$
    \par $while(i \leq m) do$
    \par \hspace*{1cm} $if(\langle T \cup \lbrace e_i \rbrace \rangle _G$ nu are circuite $)$  $then$
    \par \hspace*{1.5cm} $ T \leftarrow T \cup \lbrace e_i \rbrace ;$
    \par \hspace*{1cm} $i++;$
    }

\subsection*{\fontsize{16}{30}\selectfont Subpunctul b}
{\fontsize{14}{16}\selectfont 
$\mathbf{Implicatia}$ $\mathbf{directa:}$ DRAFT
\\
\par $T$ este un arbore partial al lui $G$, deci $T$ conecteaza toate nodurile din $G$ si este un subgraf al lui $G$. Prin definitie $(c(e)\geq \alpha)$, orice muchie din $T_\alpha$ este inclusa in $G_\alpha$. Deoarece $T_\alpha$ este subarbore al lui $T$ si $G_\alpha$ este subgraf al lui $G$, toate conexiunile intre nodurile din $T_\alpha$ sunt valabile si in $G_\alpha$.         
\\
\par $\mathbf{Implicatia}$ $\mathbf{inversa:}$
\\
}    

\subsection*{\fontsize{16}{30}\selectfont Subpunctul c}
{\fontsize{14}{16}\selectfont Initial vom folosi algoritmul lui Kruskal modificat de la subpunctul a$)$ pentru a construi arborele partial de cost maxim din $G$.}

\section*{\fontsize{20}{50}\selectfont Problema 3}
\subsection*{\fontsize{16}{30}\selectfont Subpunctul a}
{\fontsize{14}{16}\selectfont 

\centerline {$ c(T|e) = \sum_{ e'\in E(T)\setminus \lbrace e \rbrace } c(e') = \sum_{ e'\in E(T) } c(e') - c(e) = c(T) - c(e)$ }
}

\subsection*{\fontsize{16}{30}\selectfont Subpunctul b}
{\fontsize{14}{16}\selectfont 
    Fie $T$ un arbore partial de cost minim al lui $G$, avand costul minim $c(T)$. Conform subpuctului $a$, prin contractarea muchiei e, obtinem $c(T|e) = c(T) - c(e)$.
    Sa presupunem prin reducere la absurd ca $T|e$ $\mathbf{N} \mathbf{U}$ este un arbore partial de cost minim al lui $G|e$. Atunci ar exista un alt arbore partial $T'|e$ a lui $G|e$, cu $c(T'|e) < c(T|e)$.
    \par Daca $T'|e$ este un arbore de cost mai mic n $G|e$, atunci putem extinde acest arbore n $G$ adaugand inapoi muchia $e$ pentru a obtine un arbore $T'$ al lui $G$. Costul acestui arbore $T'$ ar fi:
    \par \centerline { $c(T') = c(T'|e) + c(e)$ }
    \par Deoarece $c(T'|e) < c(T|e)$, rezulta ca:
    \par \centerline {$c(T'|e) < c(T|e) + c(e) = c(T) $}
    \par Aceasta contrazice ipoteza ca $T$ este un arbore partial de cost minm al lui $G$. Prin urmare, $T|e$ trebuie sa fie un arbore partial de cost minim al lui $G|e$.
    }

\end{document}
    \centerline {$ c(T|e) = \sum_{ e'\in E(T)\setminus \lbrace e \rbrace } c(e') = \sum_{ e'\in E(T) } c(e') - c(e) = c(T) - c(e)$ }